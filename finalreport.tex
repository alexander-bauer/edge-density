\documentclass[11pt]{article}

\usepackage{graphicx}
\usepackage{subfig}

\usepackage[backend=biber]{biblatex}
\addbibresource{references}

\usepackage{../cmsc491}
\fancyhead[R]{CMSC 491, Final Report}

\pagestyle{fancy}

\begin{document}

\begin{center}
    {\LARGE Rapid Clutter Recognition using Edge Detectors} \\
    {\large Alexander Bauer}
\end{center}

\section*{Abstract}

Shared living spaces can rapidly accumulate disorganized miscellaneous items, or
``clutter.'' Different individuals may have very different tolerances for
clutter in the household, and this can be a source of tension among housemates.
Using commodity cameras and processors, the methods presented here can provide a
baseline clutter detector. Results may serve as a prompt for residents to begin
picking up, or as a guide to identify areas of greatest importance.

\section*{Problem Statement}

The objective of this project was to automatically identify and locate cluttered
regions in photos taken using surveillance and smartphone cameras.  ``Clutter''
is a subjective term, but for these purposes was broadly taken to include any
small foreign or out-of-place object in a scene. For example: shoes, dishes,
tissue boxes are clutter, but furniture is not.

\section*{Related Research}

\section*{Methods}



\pagebreak
\section*{References}

\printbibliography

\end{document}
