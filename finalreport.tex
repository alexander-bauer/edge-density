\documentclass[11pt]{article}

\usepackage{graphicx}
\usepackage{subfig}

\usepackage[backend=biber,style=ieee]{biblatex}
\addbibresource{references.bib}

\usepackage{../cmsc491}
\fancyhead[R]{CMSC 491, Final Report}

\pagestyle{fancy}
\linespread{1.1}

\begin{document}

\begin{center}
    {\LARGE Rapid Clutter Recognition} \\
    {\large Alexander Bauer}
\end{center}

\section*{Introduction}

Shared living spaces can rapidly accumulate disorganized miscellaneous items, or
``clutter.'' Different individuals may have very different tolerances for
clutter in the household, and this can be a source of tension among cohabitants.
Using commodity cameras and processors, the methods presented here serve as a
rapid and automatic clutter detector. These results may be used to automatically
prompt residents to begin picking up, as a guide to identify areas of
greatest importance, or to numerically track trends in household cleanliness.

\section*{Problem Statement}

The objective of this project was to automatically identify and locate cluttered
regions in photos taken using surveillance and smartphone cameras. ``Clutter''
is a subjective term, but for these purposes was broadly taken to include any
small foreign or out-of-place object in a scene. For example: shoes, dishes,
tissue boxes are clutter, but furniture is not.

\section*{Related Research}

A wealth of research is available for the problem of object
recognition \cite{Lowe99,Besl86,Torralba03,Torralba04}, but
that is not quite the domain of this problem. Furthermore, there exist many
schemes to recognize objects of interest in cluttered
scenes \cite{Mori03,Johnson99}. Due to the apparent absence of scholarly
research available on the general recognition of clutter itself, this author is
led to believe it is a novel application.

\section*{Methods}

Providing the context necessary to identify ``out-of-place'' objects in a scene
is complicated problem in its own right, as is the generalized object
recognition capability necessary to apply them. Hence, it is necessary to use
other heuristics as an analogue for actual clutter.

There are two techniques presented here: simple corner detection, and dense edge
detection. The former serves as a primitive baseline with a high false-positive
rate that is unable to capture regions. The latter is capable of describing
entire regions which appear cluttered.

\subsection*{Corner Detection}

Full-color images are converted to grayscale, and corners are detected using the
Harris corner detector \cite{Harris88}. The OpenCV implementation
\cite{cv2:cornerHarris} is invoked with block size of $2$, kernel size of $3$,
and free parameter of $0.16$. Resulting points are dilated for display and drawn
on a new blank image.

See figure~\ref{fig:corner_01} for an example and test results. In
\ref{fig:corner_01_truth}, green regions are true positive cluttered areas, red
are false negatives, and blue are false positives.

\begin{figure}[p]
    \begin{center}
    \begin{tabular}{ccc}
        \subfloat[Original Image]{
            \includegraphics[width=0.3\hsize]{results/process/corners/occupied_01.jpg}} &
        \subfloat[Harris Corners]{
            \includegraphics[width=0.3\hsize]{results/process/corners/corners_01.jpg}} &
        \subfloat[Tests Against Ground Truth]{
            \includegraphics[width=0.3\hsize]{results/process/corners/test_results_01.jpg}
            \label{fig:corner_01_truth}
        }
    \end{tabular}
    \end{center}
    \caption{Corner Detection on Surveillance Camera Photo}
    \label{fig:corner_01}
\end{figure}

\subsection*{Dense Edge Detection}

Full-color images are converted to grayscale, and a new image is rendered by
drawing edges detected by the Canny edge detector \cite{Canny86, cv2:Canny} with
thresholds $200$ and $400$. These are drawn on a new blank image and then
blurred using a box filter \cite{McDonnell81, cv2:boxFilter} of size $30 \times
30$. This gives high results in areas with dense edges, and low results in areas
with few or no edges. These are clipped using an adaptive Gaussian threshold
\cite{Wilkinson03,cv2:adaptiveThreshold} with a box size of $101 \times 101$ and
a constant of $10$ added to the mean threshold. Surrounding contours are then found
for each region \cite{Suzuki85,cv2:findContours}, and their convex hulls
\cite{Sklansky82,cv2:convexHull} are drawn on a new blank image.

Steps are shown in order in figure~\ref{fig:dense_01}.
Subfigure~\ref{fig:dense_01_truth} shows results of this detector
against the ground truth.

\begin{figure}[p]
    \begin{center}
    \begin{tabular}{ccc}
        \subfloat[Original Image]{
            \includegraphics[width=0.3\hsize]{results/process/dense_edges/occupied_01.jpg}} &
        \subfloat[Canny Edges]{
            \includegraphics[width=0.3\hsize]{results/process/dense_edges/edges_01.jpg}} &
        \subfloat[Blurred Canny Edges]{
            \includegraphics[width=0.3\hsize]{results/process/dense_edges/blurred_01.jpg}} \\
        \subfloat[Threshold Results]{
            \includegraphics[width=0.3\hsize]{results/process/dense_edges/dense_01.jpg}} &
        \subfloat[Convex Hulls]{
            \includegraphics[width=0.3\hsize]{results/process/dense_edges/convex_01.jpg}} &
        \subfloat[Tests Against Ground Truth]{
            \includegraphics[width=0.3\hsize]{results/process/dense_edges/test_results_01.jpg}
            \label{fig:dense_01_truth}
        }
    \end{tabular}
    \end{center}
    \caption{Dense Edge Detection on Surveillance Camera Photo}
    \label{fig:dense_01}
\end{figure}

\section*{Results}

For testing these methods, two sets of images of the same scene were prepared
with different levels of clutter, and were hand-classified to provide a ground
truth. Test results are shown on each in
figures~\ref{fig:corners_test_01}-\ref{fig:dense_test_02}. Green regions are true
positives, blue regions are false positives, and red regions are false
negatives.

A numerical measure of the success of the detectors was given by a confusion
matrix. The elements are true positives, false positives, false negatives, and
true negatives, each given as a percentage of total pixels. This measure has
severe limitations; confusion matrices from different images may not be easily
compared, because they do not normalize across quantity of clutter.

\begin{center}
    {\large Corner Detector across Datasets 1 and 2} \\[1em]
\begin{tabular}{c|cc}
    & Detected Clutter & Detected Non-Clutter \\
    \hline
    Actual Clutter     & $0.012\%$ & $0.124\%$ \\
    Actual Non-Clutter & $0.027\%$ & $0.837\%$
\end{tabular}
\end{center}
\hfill

\begin{center}
    {\large Dense Edge Detector across Datasets 1 and 2} \\[1em]
\begin{tabular}{c|cc}
    & Detected Clutter & Detected Non-Clutter \\
    \hline
    Actual Clutter     & $0.027\%$ & $0.106\%$ \\
    Actual Non-Clutter & $0.055\%$ & $0.811\%$
\end{tabular}
\end{center}

These numerical results clearly indicate that although the dense edge detector
has a higher rate of false positives by area, it more correctly classifies regions of
actual clutter than the naive corner detector. 

More notably, visual results indicate that although the dense edge detector does
not cover areas of clutter, it is able to locate a small portion of most or all
cluttered regions. Hence, if false positives were suppressed, it could serve as
a sufficient classifier. This indicates that better results could be achieved
with more parameter tuning, or with extensions of this simple method.

For example, a machine learning algorithm could be trained with coordinates of
false positives in images taken by mounted security camera. Then, in future
predictions, known false positives (such as wall hangings, as appear prominently
in figure~\ref{fig:dense_test_01}) could be discarded.

These results are promising for home automation applications, particularly with
the potential for false positive suppression. Additionally, as seen in
figure~\ref{fig:comparison_outdoor}, outdoor trash detection may also be a
possibility. Areas of further study might include use of low-flying photography
drones to locate trash and litter in public places.

\begin{figure}[p]
    \begin{center}
    \begin{tabular}{ccc}
        \subfloat[]{
            \includegraphics[width=0.3\hsize]{results/test/corners/occupied_00.jpg}} &
        \subfloat[]{
            \includegraphics[width=0.3\hsize]{results/test/corners/occupied_01.jpg}} &
        \subfloat[]{
            \includegraphics[width=0.3\hsize]{results/test/corners/occupied_02.jpg}} \\
        \subfloat[]{
            \includegraphics[width=0.3\hsize]{results/test/corners/occupied_03.jpg}} &
        \subfloat[]{
            \includegraphics[width=0.3\hsize]{results/test/corners/occupied_04.jpg}} &
        \subfloat[]{
            \includegraphics[width=0.3\hsize]{results/test/corners/occupied_05.jpg}} \\
        \subfloat[]{
            \includegraphics[width=0.3\hsize]{results/test/corners/occupied_06.jpg}} &
        \subfloat[]{
            \includegraphics[width=0.3\hsize]{results/test/corners/occupied_07.jpg}} &
        \subfloat[]{
            \includegraphics[width=0.3\hsize]{results/test/corners/occupied_08.jpg}} \\
    \end{tabular}
    \end{center}
    \caption{Corners Test Results on Dataset 1}
    \label{fig:corners_test_01}
\end{figure}

\begin{figure}[p]
    \begin{center}
    \begin{tabular}{ccc}
        \subfloat[]{
            \includegraphics[width=0.3\hsize]{results/test/corners/01.jpg}} &
        \subfloat[]{
            \includegraphics[width=0.3\hsize]{results/test/corners/02.jpg}} &
        \subfloat[]{
            \includegraphics[width=0.3\hsize]{results/test/corners/03.jpg}} \\
        \subfloat[]{
            \includegraphics[width=0.3\hsize]{results/test/corners/04.jpg}} &
        \subfloat[]{
            \includegraphics[width=0.3\hsize]{results/test/corners/05.jpg}} &
        \subfloat[]{
            \includegraphics[width=0.3\hsize]{results/test/corners/06.jpg}} \\
        \subfloat[]{
            \includegraphics[width=0.3\hsize]{results/test/corners/07.jpg}} &
        \subfloat[]{
            \includegraphics[width=0.3\hsize]{results/test/corners/08.jpg}} &
        \subfloat[]{
            \includegraphics[width=0.3\hsize]{results/test/corners/09.jpg}}
    \end{tabular}
    \end{center}
    \caption{Corners Test Results on Dataset 2}
    \label{fig:corners_test_02}
\end{figure}

\begin{figure}[p]
    \begin{center}
    \begin{tabular}{ccc}
        \subfloat[]{
            \includegraphics[width=0.3\hsize]{results/test/dense/occupied_00.jpg}} &
        \subfloat[]{
            \includegraphics[width=0.3\hsize]{results/test/dense/occupied_01.jpg}} &
        \subfloat[]{
            \includegraphics[width=0.3\hsize]{results/test/dense/occupied_02.jpg}} \\
        \subfloat[]{
            \includegraphics[width=0.3\hsize]{results/test/dense/occupied_03.jpg}} &
        \subfloat[]{
            \includegraphics[width=0.3\hsize]{results/test/dense/occupied_04.jpg}} &
        \subfloat[]{
            \includegraphics[width=0.3\hsize]{results/test/dense/occupied_05.jpg}} \\
        \subfloat[]{
            \includegraphics[width=0.3\hsize]{results/test/dense/occupied_06.jpg}} &
        \subfloat[]{
            \includegraphics[width=0.3\hsize]{results/test/dense/occupied_07.jpg}} &
        \subfloat[]{
            \includegraphics[width=0.3\hsize]{results/test/dense/occupied_08.jpg}} \\
    \end{tabular}
    \end{center}
    \caption{Dense Edges Test Results on Dataset 1}
    \label{fig:dense_test_01}
\end{figure}

\begin{figure}[p]
    \begin{center}
    \begin{tabular}{ccc}
        \subfloat[]{
            \includegraphics[width=0.3\hsize]{results/test/dense/01.jpg}} &
        \subfloat[]{
            \includegraphics[width=0.3\hsize]{results/test/dense/02.jpg}} &
        \subfloat[]{
            \includegraphics[width=0.3\hsize]{results/test/dense/03.jpg}} \\
        \subfloat[]{
            \includegraphics[width=0.3\hsize]{results/test/dense/04.jpg}} &
        \subfloat[]{
            \includegraphics[width=0.3\hsize]{results/test/dense/05.jpg}} &
        \subfloat[]{
            \includegraphics[width=0.3\hsize]{results/test/dense/06.jpg}} \\
        \subfloat[]{
            \includegraphics[width=0.3\hsize]{results/test/dense/07.jpg}} &
        \subfloat[]{
            \includegraphics[width=0.3\hsize]{results/test/dense/08.jpg}} &
        \subfloat[]{
            \includegraphics[width=0.3\hsize]{results/test/dense/09.jpg}}
    \end{tabular}
    \end{center}
    \caption{Dense Edges Test Results on Dataset 2}
    \label{fig:dense_test_02}
\end{figure}

\begin{figure}[p]
    \begin{center}
    \begin{tabular}{cc}
        \subfloat[Corners]{
            \includegraphics[width=0.5\hsize]{results/test/corners/handheld_01.jpg}} &
        \subfloat[Dense Edges]{
            \includegraphics[width=0.5\hsize]{results/test/dense/handheld_01.jpg}} \\
        \subfloat[Corners]{
            \includegraphics[width=0.5\hsize]{results/test/corners/handheld_02.jpg}} &
        \subfloat[Dense Edges]{
            \includegraphics[width=0.5\hsize]{results/test/dense/handheld_02.jpg}}
    \end{tabular}
    \end{center}
    \caption{Indoor Photos with a Handheld Camera}
    \label{fig:comparison_indoor}
\end{figure}
\begin{figure}[p]
    \begin{center}
    \begin{tabular}{cc}
        \subfloat[Corners]{
            \includegraphics[width=0.5\hsize]{results/test/corners/outdoor_01.jpg}} &
        \subfloat[Dense Edges]{
            \includegraphics[width=0.5\hsize]{results/test/dense/outdoor_01.jpg}} \\
        \subfloat[Corners]{
            \includegraphics[width=0.5\hsize]{results/test/corners/outdoor_02.jpg}} &
        \subfloat[Dense Edges]{
            \includegraphics[width=0.5\hsize]{results/test/dense/outdoor_02.jpg}}
    \end{tabular}
    \end{center}
    \caption{Outdoor Photos with a Handheld Camera}
    \label{fig:comparison_outdoor}
\end{figure}

\pagebreak
\printbibliography

\end{document}
