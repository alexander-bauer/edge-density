\documentclass{article}

\usepackage{../cmsc491}
\fancyhead[R]{CMSC 491, Project Proposal}

\pagestyle{fancy}

\linespread{1.3}

\begin{document}

\begin{center}
    \LARGE Proposal: Occupancy and ``Clutter'' Detection \\
    for Home Automation
\end{center}

Computer vision can be used to leverage sensing capabilities of consumer cameras
and computer hardware for home automation purposes. Home automation is a rapidly
growing field involving applied computer science, which currently suffers from
an abundance of ``output'' hardware, such as smart thermostats and
network-connected light bulbs, but a small amount of consumer-accessible
``input'' hardware for sensing home conditions, such as occupancy, motion, and
ambient lighting.\footnote{Such sensors exist, but often involve integrated
circuitry and custom software components.}

Network- or PC-connected visible-light cameras are common in many households,
and hence may allow us to leverage computer vision for numerous sensing tasks,
including occupancy detection. I propose to study not only occupancy detection
using consumer-grade cameras (Raspberry Pi camera and security-camera
installations), but also ``clutter'' detection, to measure the degree of
miscellaneous object clutter in a living space and indicate when cleaning up is
necessary. Both applications will be run on consumer hardware suitable for
actual home use.

Occupancy is a related but distinct problem from motion detection, which is
often solved by passive infrared sensors (PIRs).\footnote{Adafruit PIR kits
    for the Raspberry Pi: \url{https://adafruit.com/products/189}.}
Occupants of a living area may not be moving, and may be strictly humanoid
(whereas PIRs may detect pets); computer vision may be able to solve this
problem. The popular open source library
OpenCV\footnote{\url{http://opencv.org}}
includes a pre-trained histogram of oriented gradients (HoG) detector which I
expect to use.

Clutter detection appears to be a novel application for computer vision. My
first approach will be to use an HoG descriptor on a training set comprised of
images of a house in various stages of clutter. This will likely require the
greatest portion of research time.

Currently, my languages of choice for the project are Python and
Julia,\footnote{\url{http://julialang.org}} a fast data-science language. They
have excellent cross-language calling support, and seem well-suited to the
application. I am also aiming to integrate my software with Idiotic, an open
source home automation
framework.\footnote{\url{https://github.com/idiotic/idiotic}} Using Python will
make it very easy to include as a module, and hence release under an open source
license for wider use.

\end{document}
